%documentclass [12pt]{article}
\documentclass[12pt, oneside]{article}
\usepackage{setspace} % for double-spacing.
\usepackage[top=2.54cm, bottom=2.54cm, left=2.54cm, right=2.54cm]{geometry}	 %
\usepackage{url}
\usepackage{mathptmx}% http://ctan.org/pkg/mathptmx
\usepackage{xifthen}%
%\doublespacing
\usepackage{lipsum}
\usepackage{breakurl}
\usepackage[breaklinks]{hyperref}
\usepackage{tocloft}
\usepackage{color}
\usepackage{array}
\usepackage{stfloats}
\usepackage[pdftex]{graphicx}
\usepackage{subcaption}
\usepackage[export]{adjustbox}
\usepackage{multirow}
\usepackage{url}
\usepackage{changepage}
\usepackage{floatrow}
\usepackage{titlesec}
\usepackage{chngcntr}
\usepackage{fancyhdr}
% -- Define code snippet configuration
\usepackage{listings}
\usepackage{color}
\usepackage{floatrow}

\usepackage{listings}
\usepackage{color}

\definecolor{dkgreen}{rgb}{0,0.6,0}
\definecolor{gray}{rgb}{0.5,0.5,0.5}
\definecolor{mauve}{rgb}{0.58,0,0.82}

\lstset{frame=tb,
  language=Java,
  aboveskip=3mm,
  belowskip=3mm,
  showstringspaces=false,
  columns=flexible,
  basicstyle={\small\ttfamily},
  numbers=none,
  numberstyle=\tiny\color{gray},
  keywordstyle=\color{blue},
  commentstyle=\color{dkgreen},
  stringstyle=\color{mauve},
  breaklines=true,
  breakatwhitespace=true,
  tabsize=3
}

\title{Using IBM's Tonality Analysis of Language and Geolocated Tweets to Map Emotional Intensity.
\\\medskip}
\author{Isaac Callison (ic2d@mtmail.mtsu.edu)\\Middle Tennessee State University}

\begin{document}
\maketitle
\nocite{*}
\newpage{}


\renewenvironment{abstract}
 {\small
  \begin{center}
  \bfseries \abstractname\vspace{-.5em}\vspace{0pt}
  \end{center}
  \list{}{
    \setlength{\leftmargin}{.8cm}%
    \setlength{\rightmargin}{\leftmargin}%
  }%
  \item\relax}
 {\endlist}

\begin{abstract}
Metadata, and extraction of relevant information from that data, is exploding.
Companies and organizations are competing at a furious rate to glean advantages
from ever expanding datasets. One such voluminous and continuously updated
dataset are tweets from the social-media giant Twitter. Tweets were collected
in real-time. Using IBM's tone analyzer, the emotional nature of these tweets
was analyzed. The results were mapped onto a Google Heatmap.
\end{abstract}



\section{Introduction}
\paragraph{}
There is a convergence of powerful technologies that allow for near-
instantaneous notification of current events using available meta-data from
social media platforms. The goal of this project was to gather and analyze
tweets for emotive tonality, then display this on a Google "heatmap" of emotive
intensity.

To this end, three separate technologies that are interdependent were utilized
for this project. First, using Twitter's developer API, geolocated tweets were be collected from a specific region and radius, then pre-processed to extract
latitude, longitude and text. Secondly, using IBM's Watson, the tweets were
assesed for emotive tonality. Finally, a Google map is displayed with a
heatmap layer graphing the intensity of these emotions.


\section{Background}
\paragraph{}
As stated above, there were several interdependent moving parts with this
project. With regards to platform, a Jupyter Notebook was used for this
project to pull in modules, access API's, and make GET and POST requests.

The datasets for the natural language processing came from Twitter. The
Twitter API allows for the triangulation of geolocated tweets\cite{TwitterGeo}.
However, this feature was culled in March of 2019. In response, other methods
were used to triangulage tweets.

Using developer authentication, and a GET request with certain location
parameters, one can obtain a list of current tweets within a search radius in
JSON format. A great deal of meta-data is returned in this format, but from
these tweets one can glean a myriad of data, including the latitude and
longitude of the tweet.

The second part of this project was natural language processing with IBM's
Watson using tonality analysis.
IBM has a cloud computing program with various machine learning
capabilities\cite{IBM}, one of which is tonality analysis. The natural language
tonality processing that Watson offers can, among other things, extract emotion
from a corpus. In this specific case, a variety of emotions, including
happiness, sadness, frustration, excitement, etc., and the intensity of that
specific emotion, were derived at the sentence and document level.

As a graphical representation of the data collected, a heatmap was used. A heatmap overlay is a feature offered by Google Maps. It can create a
visualization to depict the intensity of data at a range of geographical points.
This is good when you have lots of data points of varying magnitude.
When the Heatmap Layer is enabled, a colored overlay will appear on top of the
map. By default, areas of higher intensity will be colored red, and areas of
lower intensity will appear green\cite{Google}.

\section{Research Method}
 The consoles and APIs of Twitter, IBM's Watson, and Google were
 accessed. The three aformentioned services required developer accounts to be
 used. Twitter's API required an application and pre-approval. These
 credentials were secured. Postman, a REST API testing software was
 used to experiment with Twitter's API\cite{Postman}.

 A Jupyter Notebook was used to make GET requests of Twitter. The
 data was processed to obtain the text of the actual tweet and the locaton of
 the tweet. What was initially envisioned was a fine grain heatmap showing data
 across any number of cities. Unfortunately, the fine-grain location data of
 each tweet was removed by Twitter in mid-2019. As a result, tweets could be
 pulled from a specific region, but the exact latitude and longitude of that
 tweet was not discerable. The method was modified from the original thusly...


 As a result of the modified method, the Tweets pulled in by the Notebook were
 still run through the IBM tone analyzer after significatn pre-processing.IBM provides a Watson Software
 Development Kit for integration into Python and Jupyter. The text from the
 obtained tweets will be passed into Watson for tonality processing.

 Display a heatmap of emotion in a certain region based on the intensity
 of the selected emotional state. A Google
 Map will display a heatmap layer with the attendant emotion and intensity. For
 instance, in areas of a region where the tweets have low sadness, green shading
 will predominate, changing to red as the intensity of that emotion increases


\section{Results and Analysis}


\section{Conclusion and Future Work}
We live in an age of ever expanding meta-data. As this increases, humanity will
seek to harness this data through new technologies, for better or worse. In
this case, Twitter data can potentially be used to inform and improve the lives
of everyday citizens. If the above can be implemented, one could graph
the emotional intensity of differing regions based on twitter content.

\newpage{}

\bibliographystyle{myplain}
%\bibliographystyle{IEEEannot}
\bibliography{annot}

\newpage{}
\appendix{}
\section{Coding Environment}
All code was run on Ubuntu 18.04 LTS. Anaconda was installed and Jupyter
Notebooks were utilized. All code was written in Python 3. There was some
difficulty in getting Google Maps to display in JupyterLab. Several modules
were required to run the various APIs utilized in this project. The following process was used to install Anaconda and the modules used:

\begin{figure}[H]
\begin{lstlisting}


// From a bash command prompt:
curl -O https://repo.anaconda.com/archive/Anaconda3-5.2.0-Linux-x86_64.sh
bash Anaconda3-5.2.0-Linux-x86_64.sh


// After installation of Anaconda, these modules should be installed:
pip install ibm_watson
pip install tweepy
pip install geopy

// For gmaps:
jupyter nbextension enable --py --sys-prefix widgetsnbextension
pip install gmaps
jupyter nbextension enable --py --sys-prefix gmaps

\end{lstlisting}
\caption{Anaconda \& Module Setup}\
\label{fig:code}
\end{figure}

\end{document}
