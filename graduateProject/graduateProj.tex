%documentclass [12pt]{article}
\documentclass[12pt, oneside]{article}
\usepackage{setspace} % for double-spacing.
\usepackage[top=2.54cm, bottom=2.54cm, left=2.54cm, right=2.54cm]{geometry}	 %
\usepackage{url}

\title{Using IBM's Tonality Analysis of Language and Geolocated Tweets to Map Emotional Intensity.
\\\medskip}
\author{Isaac Callison (ic2d@mtmail.mtsu.edu)\\Middle Tennessee State University}

\begin{document}
\maketitle
\nocite{*}
\newpage{}


\renewenvironment{abstract}
 {\small
  \begin{center}
  \bfseries \abstractname\vspace{-.5em}\vspace{0pt}
  \end{center}
  \list{}{
    \setlength{\leftmargin}{.8cm}%
    \setlength{\rightmargin}{\leftmargin}%
  }%
  \item\relax}
 {\endlist}

\begin{abstract}
Blockchain technology is a burgeoning technology with massive potential to
disrupt entrenched technological modes in a variety of industries.
Current blockchain implementations manifest across desktop and cloud computing.
Little research has been invested into low-powered devices, devices often
collectively and colloquially referred to as IoT or Internet of Things. Several
types of low-level and low-powered devices of various architectures are
to be investigated. The most basic and low-level device that can still run a
recognizable private blockchain will be chosen. Several of these devices will
be run in a peer-to-peer network in an attempt to track an asset via RFID.
Early experimentation has indicated that an esp8266, a low-cost WiFi enabled
microcontroller, may have the right balance of limitations and features to
effectuate such a blockchain. Low-cost immutable recordation of asset movement
could be a boon for various types of organizations in the private and public
sector.
\end{abstract}



\section{Introduction}
\paragraph{}
There is a convergence of powerful technologies that allow for near-
instantaneous notification of current events using available meta-data from
social media platforms. The goal of this project is to gather and analyze
tweets for emotive tonality, then display this on a Google "heatmap" of emotive
intensity.

There are three separate technologies that will be interdependent in this
project. First, using Twitter's developer API, geolocated tweets will be
collected from a specific region and radius, then pre-processed to extract
latitude, longitude and text. Secondly, using IBM's Watson, the tweets will be
assesed for emotive tonality. Finally, a Google map will be displayed with a
heatmap layer graphing the intensity of these emotions.


\section{Background}
\paragraph{}
As stated above, there will be several interdependent moving parts with this
project. With regards to platform, a Jupyter Notebook will be used for this
project to pull in modules, access API's, and make GET and POST requests.
The datasets for the natural language processing will come from Twitter. The
Twitter API allows for the triangulation of geolocated tweets\cite{TwitterGeo}.
Using developer authentication, and a GET request with certain location
parameters, one can obtain a list of current tweets within a search radius in
JSON format. A great deal of meta-data is returned in this format, but from
these tweets one can glean a myriad of data, including the latitude and
longitude of the tweet.

The second part of this project is natural language processing with IBM's
Watson using tonality analysis.
IBM has a cloud computing program with various machine learning
capabilities\cite{IBM}, one of which is tonality analysis. The natural language
tonality processing that Watson offers can, among other things, extract emotion
from a corpus. In this specific case, a variety of emotions, including happiness, sadness, frustration, excitement, etc., and the intensity of that specific emotion, can be
derived at the sentence and document level.

A heatmap is an overlay feature offered by Google Maps. It can create a
visualization to depict the intensity of data at a range of geographical points.
This is good when you have lots of data points of varying magnitude. When the
Heatmap Layer is enabled, a colored overlay will appear on top of the map. By
default, areas of higher intensity will be colored red, and areas of lower
intensity will appear green\cite{Google}.

\section{Research Method}
\begin{itemize}
 \item Access the developer consoles and APIs of Twitter, IBM's Watson, and Google Maps to see if the three technology giants can mesh.

 \item Create a Jupyter Notebook that pulls in geolocated tweets that IBM's
 Watson can analyze for emotive tonality.

 \item Display a heatmap of emotion in a certain region based on the intensity
 of the selected emotional state.

\end{itemize}

\section{Results and Analysis}
A few steps have already been implemented to see this project to fruition. The
three aformentioned services require developer accounts to be used.
Twitter's API required an application and pre-approval. These credentials have
been secured. Postman, a REST API testing software has been used to experiment
with Twitter's API\cite{Postman}. Further, a skeleton Jupyter Notebook has
been created and uploaded to Github\cite{Git}. If you would like to view this code please visit the aformentioned Github link. To date, this Notebook just
pulls in and demonstrates the Google heatmap layer with some sample earthquake
data and also does some basic tonality analysis on a few sentences.


The Jupyter Notebook is the nexus of this project. Moving forward attempts will
be made to integrate calls to the twitter API using the credentials referenced
earlier. Passing in the latitude and longitude of a specific region should
yield tweets that can then be parsed. IBM provides a Watson Software
Development Kit for integration into Python and Jupyter. The text from the
obtained tweets will be passed into Watson for tonality processing. A Google
Map will display a heatmap layer with the attendant emotion and intensity. For
instance, in areas of a region where the tweets have low sadness, green shading
will predominate, changing to red as the intensity of that emotion increases.

\section{Conclusion and Future Work}
We live in an age of ever expanding meta-data. As this increases, humanity will
seek to harness this data through new technologies, for better or worse. In
this case, Twitter data can potentially be used to inform and improve the lives
of everyday citizens. If the above can be implemented, one could graph
the emotional intensity of differing regions based on twitter content.

\newpage{}

\bibliographystyle{myplain}
%\bibliographystyle{IEEEannot}
\bibliography{annot}

\newpage{}
\section{Appendix}



\end{document}
