%documentclass [12pt]{article}
\documentclass[12pt, oneside]{article}
\usepackage{setspace} % for double-spacing.
\usepackage[top=2.54cm, bottom=2.54cm, left=2.54cm, right=2.54cm]{geometry}	 %
\usepackage{url}

\title{Situational Awareness: Using Twitter Geolocated Tweets and IBM Watson on a Mobile Device.
\\\medskip Research Proposal}
\author{Isaac Callison (ic2d@mtmail.mtsu.edu)\\Middle Tennessee State University}

\begin{document}
\maketitle
\nocite{*}
\newpage{}

\section{Introduction}
\paragraph{}
There is a convergence of powerful technologies that allow for near-
instantaneous notification of current events using available meta-data.
The goal of this project is to gather tweets from a specific radius, analyze the tweets for emotive tonality, and display on a Google map a "heatmap" of emotive intensity.

There are three separate technologies that will be interdependent in this
project. First, using Twitter's developer API, geolocated tweets will be
collected from a specific radius and pre-processed. Secondly, using IBM's Watson, the tweets will be assesed for tonality. Finally a Google map will be dislayed with a heatmap layer.


\section{Specific Aims}
\begin{itemize}
 \item Access the developer consoles and APIs of Twitter and IBM Watson to see
 if the two technology giants can mesh.

 \item Create an Jupyter Notebook that pulls in geolocated tweets that IBM's Watson can analyze for emotive tonality.

 \item Display a heatmap of emotion in a certain area based on intesity of the selecteed emotional state.

\end{itemize}

\section{Background}
\paragraph{}
There will be several interdependent moving parts with this project.
A Jupyter Notebook will be the Platform for this project.
The datasets for the natural language processing will come from Twitter. The
Twitter API allows for the triangulation of geolocated tweets\cite{TwitterGeo}.
Using developer authentication, and a GET request with certain location parameters, one can obtain a list of current tweets within a search radius in JSON format. From these tweets one can glean a myriad of data, including the lattitude and longitude of the tweet.

The second part of this project is natural language processing with IBM's Watson using tonality analysis.
IBM has an easily accessible cloud computing program with various machine learning capabilities\cite{IBM}. The natural language processing that Watson offers can, among other things, extract emotion and sentiment from a corpus.


A heatmap is a visualization used to depict the intensity of data at
geographical points. When the Heatmap Layer is enabled, a colored overlay will
appear on top of the map. By default, areas of higher intensity will be colored
red, and areas of lower intensity will appear green.


\section{Preliminary Results}
A few steps have already been implemented to see this project to fruition. Both
IBM's cloud machine learning service and Twitter's API require developer
accounts and some pre-approval. These credentials have been secured.
Postman, a REST API testing software has been used to experiment with Twitter's
API\cite{Postman}.
Further, a skeleton android application has been created and uploaded to Github. To date, this application just obtains and tracks the user's current position.

\section{Work Plan}
The mobile application is the nexus of this project. Moving foward attempts will be made to integrate calls to the twitter API using the credentials referenced earlier. Passing in the user's lattitude and longitude should yield tweets that can then be parsed. IBM provides a Watson Software Development Kit for integration into the Android Platform. The text from the obtained tweets will be passed into Watson for natural language processing.


\section{Broader Impacts}
We live in an age of ever expanding meta-data. As this increases, humanity will seek to harness this data through new technologies, for better or worse. In this case, Twitter data can potentially be used to inform and improve the lives of everyday citizens.
\newpage{}
\bibliographystyle{myplain}
%\bibliographystyle{IEEEannot}
\bibliography{annot}

\end{document}
