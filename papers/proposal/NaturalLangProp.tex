%documentclass [12pt]{article}
\documentclass[12pt, oneside]{article}
\usepackage{setspace} % for double-spacing.
\usepackage[top=2.54cm, bottom=2.54cm, left=2.54cm, right=2.54cm]{geometry}	 %
\usepackage{url}

\title{Situational Awareness: Using Twitter Geolocated Tweets and IBM Watson on a Mobile Device.
\\\medskip Research Proposal}
\author{Isaac Callison (ic2d@mtmail.mtsu.edu)\\Middle Tennessee State University}

\begin{document}
\maketitle
\nocite{*}
\newpage{}

\section{Introduction}
\paragraph{}
What situations are you walking into? There is a convergence of powerful
technologies that allow for near instantaneous notification of current events.
Can a mobile device leverage geolocated tweets from Twtter, and natural language analysis from
IBM's Watson, to produce real-time notification of potentially hazardous
situations? That is what this reasearch proposes to elucidate.

\section{Specific Aims}
\begin{itemize}
 \item Access the developer consoles and APIs of Twitter and IBM Watson to see
 if the two technology giants can mesh.

 \item Create an android app that pulls in geolocated tweets that IBM's Watson
 can analyze for sentiment and emotion.

 \item Show a user's current location on a mobile device in real-time, and potentially dangerous proximal locations, in a google map on a mobile device.

\end{itemize}

\section{Background}
\paragraph{}
There will be several interdependant moving parts with this project. The datasets for the natural language processing will come from Twitter. The
Twitter API allows for the triangulation of geolocated tweets\cite{TwitterGeo}.
Using developer authentication, and a GET request with certain location parameters, one can obtain a list of current tweets within a search radius in JSON format. From these tweets one can glean a myriad of data.

The android application, whimsically named "DangerFloof" after a dangerous animal, will track the user's longitude and lattitude in real-time. A mobile phone is useful in this regard because of the built-in sensor data that can accurately locate the position of a user. Using this data, tweets will be obtained from a given radius. The tweets will be pre-processed to extract the most important data.

The last part of this project is natural language processing with IBM's Watson.
IBM has an easily accessible cloud computing program with various machine learning capabilities\cite{IBM}. The natural language processing that Watson offers can, among other things, extract emotion and sentiment from a corpus.

The code for the Android Application is located on Github. While it is not
eligible to be uploaded to the Play Store yet, it may be tweaked and uploaded
in the future\cite{Git}. It is available for cloning from Github and can be run on an emulator or uploaded to a physical device.

\section{Preliminary Results}
A few steps have already been implemented to see this project to fruition. Both
IBM's cloud machine learning service and Twitter's API require developer
accounts and some pre-approval. These credentials have been secured.
Postman, a REST API testing software has been used to experiment with Twitter's
API\cite{Postman}.
Further, a skeleton android application has been created and uploaded to Github. To date, this application just obtains and tracks the user's current position.

\section{Work Plan}
The mobile application is the nexus of this project. Moving foward attempts will be made to integrate calls to the twitter API using the credentials referenced earlier. Passing in the user's lattitude and longitude should yield tweets that can then be parsed. IBM provides a Watson Software Development Kit for integration into the Android Platform. The text from the obtained tweets will be passed into Watson for natural language processing.


\section{Broader Impacts}
We live in an age of ever expanding meta-data. As this increases, humanity will seek to harness this data through new technologies, for better or worse. In this case Twitter data can potentially be used to inform and improve the lives of everyday citizens.

\bibliographystyle{myplain}
%\bibliographystyle{IEEEannot}
\bibliography{annot}

\end{document}
