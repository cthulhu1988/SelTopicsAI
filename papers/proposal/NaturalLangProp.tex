%documentclass [12pt]{article}
\documentclass[12pt, oneside]{article}
\usepackage{setspace} % for double-spacing.
\usepackage[top=2.54cm, bottom=2.54cm, left=2.54cm, right=2.54cm]{geometry}	 % for margin requirements.
\title{Situational Awareness: Using Twitter Geolocated Tweets and IBM Watson on a Mobile Device.
\\\medskip Research Proposal}
\author{Isaac Callison (ic2d@mtmail.mtsu.edu)\\Middle Tennessee State University}

\begin{document}
\maketitle
\nocite{*}

\section{Introduction}
\paragraph{}
What situations are you walking into? There is a convergence of powerful
technologies that allow for near instantaneous notification of current events.
Can a mobile device leverage geolocateed tweets from Twtter, and analysis from
IBM's Watson, to produce real-time notification of potentially hazardous
situations? That is what this reasearch proposes to elucidate.

\section{Specific Aims}
\begin{itemize}
 \item Access the developer consoles and APIs of Twitter and IBM Watson to see
 if the two technology giants can mesh

 \item Create an android app that pulls in geolocateed tweets that IBM's Watson
 can analyze for sentiment and emotion.

 \item Show a user's current location, and potentially dangerous proximal
 locations, in a google map on a mobile device.

\end{itemize}

\section{Background}
\paragraph{}
There will be several interdependant moving parts with this project. First, the
Twitter API allows for the triangulation of geolocated tweets\cite{TwitterGeo}.
Using developer authentication, and a GET request with certain location parameters, one can obtain a list of current tweets within a search radius.

The application will track the user's longitude and lattitude in real-time. Using this data, tweets will be obtained from a given radius around the user. The last part of this project is natural language processing with IBM's Watson. 

The code for the Android Application is located on Github. While it is not
eligible to be uploaded to the Play Store yet, it may be tweaked and uploaded
in the future\cite{Git}.

\section{Preliminary Results}

\section{Work Plan}


\subsection{Aim 1: Hardware Implementation}

\subsubsection{Objective}

\begin{itemize}
 \item Method:
 \item Method:
\end{itemize}

\subsubsection{Objective}

\begin{itemize}
 \item Method:
 \item Method:
\end{itemize}

\subsection{Aim 2: Node Swarm}

\subsubsection{Objective}
\begin{itemize}
 \item Method:
 Aim.
\end{itemize}

\subsubsection{Objective}
\begin{itemize}
 \item Method:
\end{itemize}

\subsubsection{Objective}
\begin{itemize}
 \item Method:
\end{itemize}


\section{Broader Impacts}

\bibliographystyle{IEEEannot}
\bibliography{annot}

\end{document}
